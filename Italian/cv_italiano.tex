\documentclass[margin, 10pt]{res} 
\usepackage{helvet}
\usepackage{framed}
\usepackage{tabstackengine}
\usepackage{xcolor}
\usepackage{hyperref}
\usepackage{longtable}
\hypersetup{
  colorlinks=true,
  linkcolor=blue!50!red,
  urlcolor=black!50!red
}

\setlength{\textwidth}{5.1in}

\begin{document}

%----------------------------------------------------------------------------------------
%	NAME AND ADDRESS SECTION
%----------------------------------------------------------------------------------------

\moveleft.5\hoffset\centerline{\large\bf Enrico Buratto, Curriculum Vitae}
\moveleft\hoffset\vbox{\hrule width\resumewidth height 1pt}\smallskip
\moveleft.5\hoffset\centerline{Via Tiziano Vecellio 3}
\moveleft.5\hoffset\centerline{31030 Dosson di Casier (TV), Italy}
\moveleft.5\hoffset\centerline{(+39) 346 832 2431}
\moveleft.5\hoffset\centerline{enrico.buratto.pub96@gmail.com}
\moveleft.5\hoffset\centerline{www.linkedin.com/in/enrico-buratto-04104b151}

\begin{resume}

%----------------------------------------------------------------------------------------
%	Technology SKILLS SECTION
%----------------------------------------------------------------------------------------

\section{Competenze tecnologiche} 

\textbf{Linguaggi di programmazione: } TypeScript \textit{(livello avanzato)}, JavaScript \textit{(livello avanzato)}, C++ \textit{(livello avanzato)}, Python \textit{(livello intermedio)}, Java \textit{(livello base)}.

\textbf{Framework: } Ionic \textit{(livello avanzato)}, Angular \textit{(livello intermedio)}, Qt \textit{(livello intermedio)}, React \textit{(livello base)}.

\textbf{Altri linguaggi: } AMPL, Matlab, R, HTML5, CSS3, SQL, \LaTeX.

\textbf{Progettazione e gestione database: } PhpMyAdmin, MySQL.

\textbf{Sistemi VCS e CI/CD: } Git, Github, Gitlab, TravisCI.

\textbf{Sistemi operativi: } GNU/Linux \textit{(OS prinicipale, particolarmente Arch Linux e Debian)}, Microsoft Windows, Apple macOS.

\textbf{Suite Microsoft Office e alternative Open Source.}

%----------------------------------------------------------------------------------------
%	EDUCATION SECTION
%---------------------------------------------------------------------------------------- 

\section{Istruzione}
\textbf{Informatica, Laurea Magistrale} \hfill Gennaio 2021 - oggi \\
\textit{Università degli Studi di Padova} \\ \\
\textbf{Informatica, Laurea Triennale} \hfill Ottobre 2016 - Dicembre 2020 \\
Voto finale: 98/110 \\
\textit{Università degli Studi di Padova} \\ \\
\textbf{Fisica, Laurea Triennale} \hfill Ottobre 2015 - Giugno 2016 \\
\textit{Università degli Studi di Padova} \\ \\
\textbf{Liceo Scientifico} \hfill Settembre 2010 - Giugno 2015 \\
\textbf{Opzione Scienze Applicate} \\
Voto finale: 77/100 \\
\textit{Liceo Scientifico "Leonardo Da Vinci" - Treviso} 

%----------------------------------------------------------------------------------------
%	PROJECTS SECTION
%---------------------------------------------------------------------------------------- 

\section{Progetti di \\ formazione ed \\ extracurricolari}

\textbf{Predire in Grafana} \hfill Novembre 2018 - Febbraio 2019 \\
\textit{Progetto di Ingegneria del Software} \\
Sviluppo di un sistema di prevenzione delle congestioni di grandi sistemi di calcolo tramite tecniche di machine learning quali regressione lineare e support vector machines. Il progetto, sviluppato durante il corso di Ingegneria del Software della laurea triennale, è stato proposto da Zucchetti, azienda leader del settore ICT. Repository Github a \href{https://github.com/CoffeeCodeSWE/swe-predire-in-grafana}{questo link}. \\ \\
\textbf{Progetto di Tecnologie Web} \hfill Novembre 2018 - Febbraio 2019 \\
Progettazione e realizzazione di un sito Web in HTML, CSS, PHP, JavaScript e SQL. Repository Github a \href{https://github.com/enricobu96/TecWebUNIPD}{questo link}. \\ \\
\textbf{Capture the Flag} \hfill Gennaio 2018 - Aprile 2018 \\
Partecipazione a una serie di eventi focalizzati sulla \textit{computer security}, tenuto dal \href{https://spritz.math.unipd.it/}{gruppo SPRITZ}. \\
\textit{Università degli Studi di Padova} \\ \\
\textbf{Progetto di Basi di Dati} \hfill Novembre 2017 - Febbraio 2018 \\
Progetto di gruppo consistente nella progettazione e successiva implementazione di un database di un social network. Repository Github a \href{https://github.com/enricobu96/DB1718}{questo link}. \\ \\
\textbf{Partecipazione al concorso "Zero Robotics"} \hfill 2013-2014 \\
Competizione indetta dall'Ente Spaziale Europeo (ESA) consistente nella programmazione di software per \href{https://www.esa.int/Science_Exploration/Human_and_Robotic_Exploration/Education/Robot_Spheres_in_zero-gravity_action}{Spheres}.

%----------------------------------------------------------------------------------------
%	OBJECTIVES AND INTERESTS SECTION
%---------------------------------------------------------------------------------------- 

\section{Interessi}
Sviluppo di pagine e applicazioni web, sia dal lato frontend che dal lato backend, con utilizzo di diversi \textit{framework} all'avanguardia come Angular e React. \\
Sviluppo di applicativi mobile, principalmente tramite framework per linguaggio TypeScript come Ionic. \\
Progettazione e implementazione di \textit{computer networks}. \\
Progettazione e implementazione di sistemi avanzati tramite strumenti tipici del \textit{Machine Learning}. \\
Analisi dati avanzata tramite strumenti informatici.



%----------------------------------------------------------------------------------------
%	PROFESSIONAL EXPERIENCE SECTION
%----------------------------------------------------------------------------------------
 
\section{Esperienza \\ Lavorativa}

\textbf{Stagista} \hfill Settembre 2020 - Ottobre 2020 \\
\textit{Sync Lab s.r.l.} \\
Tirocinio universitario per il progetto "SyncTrace", iniziativa di Sync Lab per la prevenzione dei contagi nell'ambito della malattia Covid-19 


\textbf{Internet Ads Assessor} \hfill Ottobre 2019 - oggi \\
\textit{lionbridge.com} \\
Valutatore di Google Ads.

\textbf{Copywriter} \hfill Gennaio 2016 - Gennaio 2019 \\
\textit{o2o.it}
\begin{itemize}
\item Scrittura di articoli interenti a tecnologia e informatica;
\item Scrittura di articoli generici.
\end{itemize} 


%----------------------------------------------------------------------------------------
%	COMMUNITY SERVICE SECTION
%---------------------------------------------------------------------------------------- 

\section{Volontariato}

\textbf{Capo Scout} \hfill Febbraio 2015 - oggi \\
\textit{AGESCI Gruppo Treviso 3 - Branca Lupetti e Coccinelle} \\ \\
\textbf{Barista volontario} \hfill 2012 - oggi \\
\textit{"Fiera 4 Passi", Treviso} \\ \\
\textbf{Volontariato con persone senza fissa dimora} \hfill 2013 - 2014 \\
\textit{"Emergenza freddo", cooperativa "La Esse Treviso"} \\ \\
\textbf{Volontariato con anziani} \hfill September 2012 - June 2013 \\
\textit{Casa di riposo "Casa Mia", Dosson di Casier (TV)} 


%----------------------------------------------------------------------------------------
%	LANGUAGES SECTION
%---------------------------------------------------------------------------------------- 



\section{Lingue}
\textbf{Italiano: } Madre lingua. \\
\textbf{Inglese: } Conoscenza professionale completa, certificazione IELTS C1.\\ID credenziale 21IT001504BURE010A.\\

\end{resume}
\end{document}